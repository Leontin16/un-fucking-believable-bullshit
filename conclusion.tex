\section*{Заключение}
\addcontentsline{toc}{section}{Заключение}

В ходе выполнения лабораторной работы было проведено экспериментальное сравнение двух реализаций алгоритма Дейкстры: с использованием 15-арной кучи (алгоритм A) и с использованием массивов (алгоритм B).

Основные результаты исследования:

\begin{enumerate}
	\item \textbf{Теоретический анализ} подтвердил асимптотическое преимущество реализации с d-кучей для разреженных графов: $O((n+m)\log_d n)$ против $O(n^2 + m)$ у реализации с массивами.
	
	\item \textbf{Экспериментальные результаты} показали, что:
	\begin{itemize}
		\item Для разреженных графов ($m = O(n)$) алгоритм A существенно быстрее (в 5-10 раз при $n = 10000$)
		\item Для плотных графов ($m = O(n^2)$) оба алгоритма демонстрируют сравнимую производительность
		\item Алгоритм A лучше масштабируется при увеличении количества вершин
		\item Диапазон весов ребер не оказывает значительного влияния на производительность
	\end{itemize}
	
	\item \textbf{Практические рекомендации} по выбору реализации:
	\begin{itemize}
		\item Для разреженных графов (социальные сети, дорожные карты) следует использовать реализацию с d-кучей
		\item Для плотных графов (полные или почти полные графы) можно использовать любую реализацию
		\item При работе с графами неизвестной структуры предпочтительна реализация с кучей
	\end{itemize}
\end{enumerate}

Проведенные эксперименты подтвердили теоретические оценки сложности и позволили сделать выводы о практической применимости различных реализаций алгоритма Дейкстры. Реализация с использованием 15-арной кучи показала себя как более универсальное и эффективное решение для большинства практических задач.

\textbf{Перспективы дальнейших исследований:}
\begin{itemize}
	\item Сравнение с другими приоритетными очередями (биномиальные кучи, фибоначчиевы кучи)
	\item Исследование влияния степени d в d-куче на производительность
	\item Анализ производительности на реальных графах из различных предметных областей
\end{itemize}