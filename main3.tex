\section{Эксперименты 3.3 и 3.4: Дополнительные исследования}
\label{sec:exp_m_r}

\subsection{Эксперимент 3.3: Зависимость от количества ребер при фиксированном n}

\textbf{Цель:} исследовать влияние количества ребер на производительность при фиксированном количестве вершин.

\textbf{Параметры:}
\begin{itemize}
	\item $n = 10001$ (фиксировано)
	\item $m$: от 0 до 1,000,000 с шагом 100,000
	\item Веса ребер: случайные в диапазоне $[1, 10^6]$
\end{itemize}

\begin{figure}[H]
\centering
\includegraphics[width=0.8\textwidth]{picture/result_3_3.png}
\caption{Результаты эксперимента 3.3}
\label{fig:exp3_3}
\end{figure}

\textbf{Наблюдения:}
\begin{enumerate}
	\item При $m = 0$ (граф без ребер):
	\begin{itemize}
		\item Оба алгоритма работают практически мгновенно
		\item Алгоритм A имеет небольшие накладные расходы на инициализацию кучи
	\end{itemize}
	
	\item При увеличении $m$:
	\begin{itemize}
		\item Время выполнения алгоритма B растет линейно с $m$
		\item Время выполнения алгоритма A растет медленнее благодаря эффективной структуре данных
		\item При $m = 10^6$ алгоритм A быстрее в 3-4 раза
	\end{itemize}
	
	\item Переломный момент:
	\begin{itemize}
		\item При $m < 200,000$ разница в производительности незначительна
		\item При $m > 500,000$ преимущество алгоритма A становится существенным
	\end{itemize}
\end{enumerate}

\subsection{Эксперимент 3.4: Зависимость от диапазона весов ребер}

\textbf{Цель:} исследовать влияние диапазона весов ребер на производительность.

\textbf{Параметры:}
\begin{itemize}
	\item $n = 10001$ (фиксировано)
	\item $r$ (максимальный вес): от 1 до 200 с шагом 1
	\item Два типа графов:
	\begin{itemize}
		\item Плотный: $m \approx n^2$
		\item Разреженный: $m \approx 1000n$
	\end{itemize}
	\item Минимальный вес: $q = 1$
\end{itemize}

\begin{figure}[H]
\centering
\includegraphics[width=0.9\textwidth]{picture/result_3_4.png}
\caption{Результаты эксперимента 3.4}
\label{fig:exp3_4}
\end{figure}

\textbf{Наблюдения:}
\begin{enumerate}
	\item Для плотных графов (рис. \ref{fig:exp3_4}, слева):
	\begin{itemize}
		\item Диапазон весов практически не влияет на производительность
		\item Оба алгоритма демонстрируют стабильное время выполнения
		\item Алгоритм A немного быстрее (на 10-15\%)
	\end{itemize}
	
	\item Для разреженных графов (рис. \ref{fig:exp3_4}, справа):
	\begin{itemize}
		\item Наблюдается некоторая вариативность времени выполнения
		\item Алгоритм A значительно быстрее (в 4-6 раз)
		\item Диапазон весов не оказывает систематического влияния на производительность
	\end{itemize}
	
	\item Общий вывод:
	\begin{itemize}
		\item Диапазон весов ребер не является критическим фактором для производительности алгоритма Дейкстры
		\item Основное влияние оказывают структурные характеристики графа ($n$ и $m$)
		\item Реализация с кучей менее чувствительна к изменению весов
	\end{itemize}
\end{enumerate}

\subsection{Выводы по экспериментам 3.3 и 3.4}

\begin{enumerate}
	\item Количество ребер $m$ существенно влияет на производительность, особенно для алгоритма B
	\item При увеличении $m$ преимущество алгоритма A становится более выраженным
	\item Диапазон весов ребер не оказывает значительного влияния на время выполнения
	\item Реализация с 15-кучей демонстрирует стабильную производительность при различных параметрах графа
	\item Алгоритм B более чувствителен к увеличению плотности графа
\end{enumerate}