\section*{Введение}
\addcontentsline{toc}{section}{Введение}

Алгоритм Дейкстры — один из фундаментальных алгоритмов теории графов, используемый для нахождения кратчайших путей от заданной вершины до всех остальных в взвешенном графе с неотрицательными весами ребер. Эффективность этого алгоритма во многом зависит от выбранной структуры данных для хранения и извлечения вершин с минимальным расстоянием.

В данной лабораторной работе проводится сравнительный анализ двух реализаций алгоритма Дейкстры:
\begin{enumerate}
	\item \textbf{Алгоритм A} — реализация с использованием 15-арной кучи (d-кучи с $d = 15$).
	\item \textbf{Алгоритм B} — наивная реализация с использованием массивов.
\end{enumerate}

\textbf{Цель работы:} экспериментально сравнить производительность двух реализаций алгоритма Дейкстры на графах с различными характеристиками и проверить соответствие теоретической оценки сложности практическим результатам.

\textbf{Задачи исследования:}
\begin{enumerate}
	\item Реализовать алгоритм Дейкстры с использованием 15-арной кучи.
	\item Реализовать алгоритм Дейкстры с использованием массивов.
	\item Провести серию экспериментов, варьируя параметры графов:
	\begin{itemize}
		\item количество вершин $n$;
		\item количество ребер $m$;
		\item плотность графа;
		\item диапазон весов ребер.
	\end{itemize}
	\item Проанализировать полученные результаты и сделать выводы об эффективности каждой реализации в различных условиях.
\end{enumerate}

Актуальность исследования обусловлена необходимостью выбора оптимальной реализации алгоритма Дейкстры для конкретных прикладных задач, где производительность является критическим фактором.