% Раскомментить две строки в случае компилляции в pdflatex.
\usepackage[T2A]{fontenc}
\usepackage[utf8]{inputenc}
\usepackage[russian]{babel}

\usepackage{bookmark}
%\usepackage{pscyr}
%\renewcommand{\rmdefault}{ftm} % Times New Roman

%Закомментить две строчки внизу в случае компиляции pdfLatex, а не LuaLatex.
%\usepackage{fontspec}
%\setmainfont{Times New Roman}

\usepackage{cite}
\makeatletter
\renewcommand{\@biblabel}[1]{#1.}   % Заменяем библиографию с квадратных скобок на точку
%\renewcommand{\thesection}{\arabic{section}.}
%\renewcommand{\thesubsection}{\arabic{section}.\arabic{subsection}.}
\makeatother

\usepackage{hyperref} % Работа с гипперссылками
\usepackage[dvips]{graphicx}
\usepackage{float}
\graphicspath{{picture/}}
\usepackage{amssymb,amsfonts,amsmath,amsthm} % математические дополнения от АМС
\usepackage{indentfirst} % отделять первую строку раздела абзацным отступом тоже
\usepackage[usenames,dvipsnames]{color} % названия цветов
\usepackage{makecell}
\usepackage{multirow} % улучшенное форматирование таблиц
\usepackage{ulem} % подчеркивания
\renewcommand{\arraystretch}{1.5} % standart - 1. Чуть подувеличить строки, чтоб дроби входили красиво.
\linespread{1.3} % полуторный интервал
\frenchspacing

\sloppy

\usepackage[table,xcdraw]{xcolor}
\usepackage{wrapfig}
\usepackage{pgfplots}
\pgfplotsset{compat=newest}

\usepackage{geometry}
\geometry{left = 2.5cm}
\geometry{right = 1.5cm}
\geometry{top = 2cm}
\geometry{bottom = 2cm}

\renewcommand\textbullet{\ensuremath{\bullet}} % Для нормального отображения • в шрифте tnr.
\usepackage{cancel} %Для зачеркивания

%Псевдокодик
\usepackage{algorithm}
\usepackage{algpseudocode}
\floatname{algorithm}{Алгоритм}

% C++ code and listings setup
\usepackage{listings}
\usepackage{color}

% параметры стиля с полной поддержкой кириллицы и математических символов
\lstdefinestyle{mystyle}{
    basicstyle=\footnotesize\ttfamily, % Моноширинный шрифт и размер
    breakatwhitespace=false, 
    breaklines=true,
    captionpos=b,
    keepspaces=true, 
    numbers=left,
    numbersep=5pt,
    showspaces=false,
    showstringspaces=false,
    showtabs=false,                  
    tabsize=2,
    extendedchars=true,
    mathescape=true, % <-- ГЛАВНОЕ ИЗМЕНЕНИЕ: Включаем математический режим внутри кода
    % Блок для поддержки кириллицы (T2A кодировка) - БЕЗ ПУСТЫХ СТРОК
    literate={а}{{\selectfont\char224}}1 {б}{{\selectfont\char225}}1 {в}{{\selectfont\char226}}1 {г}{{\selectfont\char227}}1 {д}{{\selectfont\char228}}1 {е}{{\selectfont\char229}}1 {ё}{{\selectfont\char184}}1 {ж}{{\selectfont\char230}}1 {з}{{\selectfont\char231}}1 {и}{{\selectfont\char232}}1 {й}{{\selectfont\char233}}1 {к}{{\selectfont\char234}}1 {л}{{\selectfont\char235}}1 {м}{{\selectfont\char236}}1 {н}{{\selectfont\char237}}1 {о}{{\selectfont\char238}}1 {п}{{\selectfont\char239}}1 {р}{{\selectfont\char240}}1 {с}{{\selectfont\char241}}1 {т}{{\selectfont\char242}}1 {у}{{\selectfont\char243}}1 {ф}{{\selectfont\char244}}1 {х}{{\selectfont\char245}}1 {ц}{{\selectfont\char246}}1 {ч}{{\selectfont\char247}}1 {ш}{{\selectfont\char248}}1 {щ}{{\selectfont\char249}}1 {ъ}{{\selectfont\char250}}1 {ы}{{\selectfont\char251}}1 {ь}{{\selectfont\char252}}1 {э}{{\selectfont\char253}}1 {ю}{{\selectfont\char254}}1 {я}{{\selectfont\char255}}1 {А}{{\selectfont\char192}}1 {Б}{{\selectfont\char193}}1 {В}{{\selectfont\char194}}1 {Г}{{\selectfont\char195}}1 {Д}{{\selectfont\char196}}1 {Е}{{\selectfont\char197}}1 {Ё}{{\selectfont\char168}}1 {Ж}{{\selectfont\char198}}1 {З}{{\selectfont\char199}}1 {И}{{\selectfont\char200}}1 {Й}{{\selectfont\char201}}1 {К}{{\selectfont\char202}}1 {Л}{{\selectfont\char203}}1 {М}{{\selectfont\char204}}1 {Н}{{\selectfont\char205}}1 {О}{{\selectfont\char206}}1 {П}{{\selectfont\char207}}1 {Р}{{\selectfont\char208}}1 {С}{{\selectfont\char209}}1 {Т}{{\selectfont\char210}}1 {У}{{\selectfont\char211}}1 {Ф}{{\selectfont\char212}}1 {Х}{{\selectfont\char213}}1 {Ц}{{\selectfont\char214}}1 {Ч}{{\selectfont\char215}}1 {Ш}{{\selectfont\char216}}1 {Щ}{{\selectfont\char217}}1 {Ъ}{{\selectfont\char218}}1 {Ы}{{\selectfont\char219}}1 {Ь}{{\selectfont\char220}}1 {Э}{{\selectfont\char221}}1 {Ю}{{\selectfont\char222}}1 {Я}{{\selectfont\char223}}1
}

% применяем стиль ко всем листингам
\lstset{style=mystyle}


\newtheorem{theorem}{Теорема}[section]
\newtheorem{corollary}{Следствие}[section]
\newtheorem{lemma}{Лемма}[section]
\newtheorem{proposition}{Утверждение}[section]
\theoremstyle{definition}
\newtheorem{definition}{Определение}[section]
\newtheorem{example}{Пример}[section]
\theoremstyle{remark}
\newtheorem*{remark}{Замечание}

\newcommand{\argmin}{\operatornamewithlimits{argmin}}

\newcommand{\Mod}[1]{\ (\mathrm{mod}\ #1)}

%\usepackage[inkscapeformat=pdf]{svg} % ЗАКОММЕНТИРУЙТЕ ЭТУ СТРОКУ, если нет пакета svg